\section{Algebra}
	
	\subsection{Functional Equations}
	Probably the most important category of questions under Algebra. In some senses, functional equations are like geometry problems: there are some established tools available, but the really interesting ones still require to invent someting on the spot.\\\\
	As a starting point, \href{run:./F_Algebra/(Evan Chen) Introduction to Functional Equations.pdf}{(Evan Chen) Introduction to Functional Equations} just about covers everything for $\mathbb{R}$-functional equations, along with detailed examples. An additional example is available at  \href{run:./F_Algebra/(Adrian Tang @ Canada 2010 Summer) Functional Equations.pdf}{(Adrian Tang @ Canada 2010 Summer) Functional Equations}.\\\\
	Functions on domains other than $\mathbb{R}$ tend to be either solvable with the same techniques (and hence easier due to domain restrictions), or they may admit pathological solutions (again, arising from the domain-specific properties). The best exposition so far of this area is \href{run:./F_Algebra/(Evan Chen) Monsters.pdf}{(Evan Chen) Monsters}, which mainly focuses on the pathological solutions to Cauchy's Equation, but also includes a walkthrough on a pathological $\mathbb{Z}$ functional equation.\\\\
	In fact, sometimes functional equations on $\mathbb{N}$ tend to involve number-theoretic properties, like divisibility and number bases, but there's no single article covering that yet.\\\\
	\textbf{Current gaps} - Pseudolinearity, and NT-specific methods.
	\subsection{Inequalities}
	You should not expect to see an old-style inequality at any competition, but I still think it's fruitful for beginners to have a cursory glance through the ``standard toolbox'', perhaps at  \href{run:./F_Algebra/(Samin Riasat) Basics of Olympiad Inequalities.pdf}{(Samin Riasat) Basics of Olympiad Inequalities}, or \href{run:./F_Algebra/(David Arthur) Cauchy-Schwarz and Other Classical Inequalities.pdf}{(David Arthur) Cauchy-Schwarz and Other Classical Inequalities}\\\\
	But aside from the standard tools, it's good in general to have a good concept of:
	\begin{itemize}
	\item \textbf{convexity} - knowing when sums of convex functions attain their maximum and/or minimum. Even some facts from majorization theory (e.g. Karamata) can be helpful. See the section about Abel Summation below.
	\item \textbf{double roots} - this gives rise to the tangent line and isolated fudging methods: there's always one step where you attempt to balance the coefficients to make the double root come out. See also: \href{run:./F_Algebra/(Thomas Mildorf @ MOP 2011) Inspired Inequalities.pdf}{(Thomas Mildorf @ MOP 2011) Inspired Inequalities [1-2]}.
	\item \textbf{smoothing} - adjusting variables to reduce the difference between the two sides of the inequality. Mixing Variables often refer to this idea as well. See also \href{run:./F_Algebra/(Thomas Mildorf @ MOP 2011) Inspired Inequalities.pdf}{(Thomas Mildorf @ MOP 2011) Inspired Inequalities [2-3]}, or \href{run:./F_Algebra/(Adrian Tang @ Canada 2010 Winter) Mixing Variables.pdf}{(Adrian Tang @ Canada 2010 Winter) Mixing Variables}.
	\item \textbf{equality cases} - the most important concept of them all. No dedicated article to this, but it's scattered everywhere.
	\end{itemize}
	If inequalities do make a come back, it's likely to be some weird sequencey stuff or something that relies more on generic bounding. The following topics may be worth a look:
	\begin{itemize}
	\item \textbf{Abel summation} - Abel (u/s)
	\item \textbf{Sum estimations} - basically, inequalities with a combinatorial flavour (!). See (u/c) for a problem set.
	\end{itemize}
	\textbf{Current gaps} - Still lacking an article that talks about smoothing on a more conceptual level. Perhaps also looking forward to better discussions for equality cases.
	\subsection{Polynomials}
	(u/c)
	\subsection{Sequences}
	This is currently quite an under-explored medium for algebra (especially since the advent of functional equations), so the only reference (= 1 example with a massive problem set) is \href{run:./F_Algebra/(Alexander Remorov @ Canada 2012 Winter) Sequences.pdf}{(Alexander Remorov @ Canada 2012 Winter) Sequences}.\\\\
	Other tangible topics include:
	\begin{itemize}
	\item \textbf{telescoping (sums/products)} - the simpler cousin of Abel. \href{run:./F_Algebra/(Gabriel Carroll @ MOP 2010) Tricky Sums and Products.pdf}{(Gabriel Carroll @ MOP 2010) Tricky Sums and Products} covers this plus a lot of other strategies for evaluating sums.
	\item \textbf{linear recurrences} - See \href{run:./F_Algebra/(WOOT 2010) Linearly Recurrent Sequences.pdf}{(WOOT 2010) Linearly Recurrent Sequences} for a quick introduction.
	\item \textbf{analysis/growth rate} - No articles yet.
	\end{itemize}
	\textbf{Current gaps} - Well, there's no conceptual breakdown of this topic at all. An overview of some bounding concepts (for seq-ineqs) or a walkthrough of some ``combinatorics-inspired'' strategies (think IVT) may be useful.
	\section{Other generic algebra}
	(u/c) completeness of reals.\\\\
	maybe future article about ``fractal'' bounds.
%