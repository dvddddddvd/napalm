\section{Geometry}
%
	\subsection{Real Basics [hmm]}
	\noindent This is for the poor sods who have never done geometry in their life. At this stage WOOT is a pretty good option, since it does try to provide a lot of motivation and advice.
	\begin{itemize}
	\item \href{run:./F_Geometry/(WOOT 2011) Circle Geometry.pdf}{(WOOT 2011) Circle Geometry} - includes similar triangles, homothety, cyclic quads, ptolemy, power, orthocenter, Simson and tangential quads
	\item \href{run:./F_Geometry/(WOOT 2012) Similar Triangles and power of a Point.pdf}{(WOOT 2012) Similar Triangles and Power of a Point} - what it says on the tin + cyclic quads and ptolemy
	\end{itemize}
%
	\subsection{Fundamentals}
	Don't get this wrong: these are not the basics. This is about the mastery of the fundamentals: the little things that matter in geometry, other than theorems and configurations.
	\begin{itemize}
	\item \href{run:./F_Geometry/(Carlos Shine @ MOP 2010) Angle Chasing.pdf}{(Carlos Shine @ MOP 2010) Angle Chasing} - shows you several ways  equal angles can appear: from the standard cyclic quads to spirals to isgonal conjugates
	\item \href{run:./F_Geometry/(Evan Chen) Directed Angles.pdf}{(Evan Chen) Directed Angles} - if you want to know why directed angles are important. Another reference is \href{run:./F_Geometry/(David Arthur @ Canada 2011 Winter) Geometry Fundamentals.pdf}{(David Arthur @ Canada 2011 Winter) Geometry Fundamentals [1-5]}.
	\item \href{run:./F_Geometry/(Carlos Shine @ MOP 2011) Constructions.pdf}{(Carlos Shine @ MOP 2011) Constructions [1]} - some strategies for constructing auxiliary points/lines. %idk if the rest of the problem set is good
	\item \href{run:./F_Geometry/(Matthew Brennan @ Canada 2014 Winter) Synthetic Geometry.pdf}{(Matthew Brennan @ Canada 2014 Winter) Synthetic Geometry [2-5]} - talks about redefining points to make things easier.
	\item \href{run:./F_Geometry/(Evan Chen @ BMC) Parallelograms.pdf}{(Evan Chen @ BMC) Parallelograms} - more construction. I like this one a lot.
	\item \href{run:./F_Geometry/(Waldemar Pompe) Quadrilaterals.pdf}{(Waldemar Pompe) Quadrilaterals.pdf} - a lot of nice problems about cyclic/tangential quadrilaterals, written by an ISL proposer.
	\item \href{run:./F_Geometry/(Richard Rusczyk @ WOOT 2008) Geometry - The Stupid Stuff Works.pdf}{(Richard Rusczyk @ WOOT 2008) Geometry - The Stupid Stuff Works} - a walkthrough of USAMO 2008 \#2, a little bit of problem solving philosophy
	\end{itemize}
	\textbf{Current gaps} -  (1) No problem set about areas, midpoints etc. (2) Also, far too little non-configurational (and basic) geometry problem sets - have to rely on Chinese books for now. (3) No article that does a good explanation of the ``mental Geogebra'' heuristic.
%
	\subsection{Power of a Point/Radical Axes}
	There are many, many articles that deal with this, but I would start with \href{run:./F_Geometry/(Yufei Zhao) Power of a Point.pdf}{(Yufei Zhao) Power of a Point}.\\\\
	Alternative references include:
	\begin{itemize}
	\item \href{run:./F_Geometry/(David Arthur @ Canada 2011 Winter) Geometry Fundamentals.pdf}{(David Arthur @ Canada 2011 Winter) Geometry Fundamentals [5-8]} - essentially a more modern version of Yufei's
	\item \href{run:./F_Geometry/(Michal Rolinek, Josef Tkadlec) Power of a Point.pdf}{(Michal Rolinek, Josef Tkadlec) Power of a Point} - illustrates how to use power to bash
	\end{itemize}
%
	\subsection{Collinearity/Concurrency}
	So everyone has heard of Ceva and Menelaus, but just in case you need a quick refresher, \href{run:./F_geometry/(WOOT 2012) Concurrency and Collinearity.pdf}{(WOOT 2012) Concurrency and Collinearity} is the best one (with a lot of other useful facts/advice for this class of problems in general).\\\\More problems also available from \href{run:./F_geometry/(Victoria Krakovna @ Canada 2010 Summer) Concurrency and Collinearity.pdf}{(Victoria Krakovna @ Canada 2010 Summer) Concurrency and Collinearity [2]}.\\\\
	Personally, I can't recall the last time I used C/M. Usually it's only when other things (like Projective/IC) can't save you from the grind.
%
	\subsection{Projective I - perspectivities, cross-ratio, harmonic(, pole/polars?)}
	Projective geometry is about interpreting synthetic properties in as projective properties, then manipulating the projective properties (usually via \textit{perspectivities}, or projection through a point). For the newcomer used to standard synthetic proofs, these concepts may be difficult to grasp at first since they are so different. \href{run:./F_geometry/(Alexander Remorov @ Canada 2010 Summer) Projective Geometry.pdf}{(Alexander Remorov @ Canada 2010 Summer) Projective Geometry [1-2]} has most of the facts down, but I love the diagrams in \href{run:./F_geometry/(Yufei Zhao @ Canada 2008 Summer) Circles.pdf}{(Yufei Zhao @ Canada 2008 Summer) Circles [3-5]}.\\\\
	Both sets also talk about poles and polars but I find that it is rarely useful; more of a conceptual tool than a problem solving one.\\\\
	\textbf{Current gaps} - No notes talking comprehensively about how to use perspectivities. Also, it seems that same-circle perspectivities are currently undocumented.
	\subsection{Projective II - Pascal/Brianchon, Desargues}
	The difference between this and Projective I (above) is mainly the use of more advanced theorems as opposed to purely perspectivities. The most important theorem in this group is \textbf{Pascal's Theorem}, and \href{run:./F_geometry/(Carl Joshua Quines) Pascal's Theorem.pdf}{(Carl Joshua Quines) Pascal's Theorem} is excellent, providing a lot of motivation and examples.\\\\
	For a quick glossary of the theorems (and some problems), you can look at both \href{run:./F_geometry/(Victoria Krakovna @ Canada 2010 Summer) Concurrency and Collinearity.pdf}{(Victoria Krakovna @ Canada 2010 Summer) Concurrency and Collinearity [3-8]} and \href{run:./F_geometry/(Alexander Remorov @ Canada 2010 Summer) Projective Geometry Continued.pdf}{(Alexander Remorov @ Canada 2010 Summer) Projective Geometry Continued [2]}.
	\textbf{Current gaps} -  The Desargue point-swapping trick.
	\subsection{Mean Geometry}
	This refers to the interpretation of points (and in general geometric objects) as algebraic quantities, but not in a coordinate geometry/complex numbers kind of way. Instead, we focus on simple algebraic relationships between points: for instance, the midpoint $M$ of a segment $AB$ can be thought of as $\frac{1}{2}A+\frac{1}{2}B$\\\\
	Of course, in this territory, the classic work to read is \href{run:./F_geometry/(Zachary Abel) Mean Geometry.pdf}{(Zachary Abel) Mean Geometry}.
%include ays's stuff?
	\subsection{Transformations}
	Since there are quite a few, let's list them down:
	\begin{itemize}
	\item \textbf{Homothety} - despite it being quite fundamental, there's no dedicated article to this (yet). But \href{run:./F_geometry/(Alexander Remorov @ Canada 2010 Summer) Projective Geometry Continued.pdf}{(Alexander Remorov @ Canada 2010 Summer) Projective Geometry Continued [2]} does have a small segment on it, but it's treated more as a configuration rather than a technique.
	\item \textbf{Isometries} - this refers to reflection, rotation and translation. (u/c) %insert MOP
	\item \textbf{Inversion} - (u/c)
	\item \textbf{Projective/Affine} - (u/c) 
	\end{itemize}
	\subsection{Configurations}
	Aside from techniques, it is good to be able to recognize recurring configurations that appear in geometry problems. Familiarity with common configurations can cut down solving time and boost your understanding of geometry, which is excellent news for newcomers, but as you become more experience you should be wary not to be over-reliant on recognition as opposed to actual solving.\\\\
	The absolute classic here is of course \href{run:./F_geometry/(Yufei Zhao @ Canada 2007 Summer) Lemmas in Euclidean Geometry.pdf}{(Yufei Zhao @ Canada 2007 Summer) Lemmas in Euclidean Geometry}, which covers:
	\begin{itemize}
	\item Symmedian lemma (intersection of tangents lie on symmedian)
	\item Diameter of incircle lemma, but please see \href{run:./F_geometry/(Evan Chen) The Incenter-Excenter Lemma.pdf}{(Evan Chen) The Incenter-Excenter Lemma} as well.
	\item Spiral similarity (briefly)
	\item ``Chicken feet'' lemma
	\item Mixtilinear incircles, but I recommend the more updated and complete \href{run:./F_geometry/(Evan Chen) Mixtilinear Incircles.pdf}{(Evan Chen) Mixtilinear Incircles} instead.
	\item Quadrilateral mixtilinear incircle lemma, a broken-down walkthrough of the (rather difficult) main lemma is available at \href{run:./F_geometry/(Yufei Zhao @ Canada 2008 Summer) Circles.pdf}{(Yufei Zhao @ Canada 2008 Summer) Circles [2-3]}
	\item Touch chord-median intersection
	\item Midline-angle bisector-touch chord concurrency
	\item Orthocenter reflection lemma
	\item $O$ and $H$ are isogonal conjugates
	\end{itemize}
	Other important configurations include:
	\begin{itemize}
	\item \textbf{Miquel/Spiral Similarity/Brokard} - \href{run:./F_geometry/(Yufei Zhao @ Canada 2009 Winter) Cyclic Quadrilaterals.pdf}{(Yufei Zhao @ Canada 2009 Winter) Cyclic Quadrilaterals} is a must-read for this. This configuration really appeared a lot a while back.
	\item \textbf{Tangential quadrilateral} - There's a little bit at \href{run:./F_Geometry/(Waldemar Pompe) Quadrilaterals.pdf}{(Waldemar Pompe) Quadrilaterals.pdf}
	\item \textbf{Anti-Steiner points} - nowhere yet.
	\item \textbf{Symmedian/Lemoine Point} - (u/c) Ricky Liu's one
	\item \textbf{Triangle centers} - (u/c) Abel's one
	\item \textbf{Casey's theorem} - For when you want to bash anything with tangent circles in the diagram.
	\end{itemize}
	\subsection{Dark Arts}
	Well, when all else fails... there's always bashing.
	Two important things to take note (for computational methods in general):
	\begin{enumerate}
	\item Please treat computational methods as an alternative rather than a shortcut. This means you should practice bashing a lot, as you would geometry as a whole, and actually memorize all the formulas.
	\item Know the limits of your selected computational method(s). You'll know what I mean if you've tried doing mixtilinear stuff with coordinate geometry. Ideally, you should be able to tell when problems/subproblems are bashable.
	\end{enumerate}
	That being said, computational methods can often be rewarding in competition scenarios, and lately there's been a \textbf{hybriding} trend: basically using synthetic observations to repeatedly reduce the questions to a bashable subproblem. This has proved to be extremely potent, even on Q3s.\\\\
	A quick overview of all computational methods:
	\begin{itemize}
	\item Coordinate geometry. Not the best, but the most introductory. This is a good model to understand the other methods with. There's no article for this, unfortunately.
	\item Trigonometry. As a general rule, once you've used cosine rule and/or angle addition formulae you're officially in a bash. No article for this, either.
	\item Length chasing. Depending on the problem and your approach, you may have trigonometric quantities or just pure length ratios (if you stuck to C/M, Stewart etc). For a quick reference, see DA (u/c)
	\item Barycentric coordinates. (u/c) Evan
	\item Complex. (u/c) Yi Sun
	\end{itemize}
